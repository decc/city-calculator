\documentclass[a4paper, twocolumn, 10pt]{article}
\usepackage{amsmath}   % for mathematical niceties
\usepackage{amsfonts}
\usepackage{biblatex}  % for citations
\usepackage{booktabs}  % for professional tables
\usepackage{hyperref}  % to make URLs clickable
%\usepackage{geometry}
% \usepackage{tabularx}
\usepackage{microtype} % Because we care
% \usepackage{color}
%
% BIBLIOGRAPHY
\addbibresource{../bibliography/cities.bib}
%
% SI UNITS
\usepackage[load-configurations=abbreviations, per-mode=symbol, mode=text]{siunitx}
\DeclareSIUnit{\year}{y}
%
% TYPOGRAPHY
\usepackage[utf8]{inputenc}          
\usepackage[T1]{fontenc}              
\usepackage[tt={monowidth}]{cfr-lm} 
%
\newcommand{\K}{\mathcal{K}}
\newcommand{\reals}{\mathbb{R}}
\title{The dimensions of physical quantities}
\author{James Geddes}
\begin{document}
\maketitle

Fix, once and for all, a set, $\K$, of vector spaces such that:
\begin{enumerate}
\item $\reals\in\K$,
\item For any $V, W\in\K$ we have $V\otimes W\in\K$; and
\item For any $V\in\K$ we have $V^*\in\K$.
\end{enumerate}

The view here is that we wish to write down physical quantities. The homes for
these physical quantities shall be the vector spaces that are the elements
of~$\K$: any physical quantity must be a vector in one of the elemets
of~$\K$. We shall call an element of $\K$ a ``quantity-kind''.

There is a natural isomorphism between $V\otimes W$ and $W\otimes V$ so we shall
consider these identical; likewise we shall treat $(V\otimes W)^*$ as identical,
under the natural isomorphism, to $V^*\otimes W^*$. 

Then $\K$ itself has a ``vector-space--like'' structure: consider the operation
of taking the tensor product as ``addition'' and taking the dual as
``multiplication by $-1$''. (Strictly speaking, it's not a vector space because
only multiplication by the integers is defined. I suppose it's a module over the
integers.) Multiplication by zero produces~$\reals$. 



 


\end{document}

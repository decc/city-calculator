\documentclass[a4paper, twocolumn, 10pt]{article}
\usepackage{amsmath}   % for mathematical niceties
\usepackage{amsfonts}
\usepackage{biblatex}  % for citations
\usepackage{booktabs}  % for professional tables
\usepackage{hyperref}  % to make URLs clickable
%\usepackage{geometry}
% \usepackage{tabularx}
\usepackage{microtype} % Because we care
% \usepackage{color}
%
% BIBLIOGRAPHY
\addbibresource{../bibliography/cities.bib}
%
% SI UNITS
\usepackage[load-configurations=abbreviations, per-mode=symbol, mode=text]{siunitx}
\DeclareSIUnit{\year}{y}
%
% TYPOGRAPHY
\usepackage[utf8]{inputenc}          
\usepackage[T1]{fontenc}              
\usepackage[tt={monowidth}]{cfr-lm} 
%
\newcommand{\physics}{\mathbb{P}}
\newcommand{\dimensions}{\mathcal{D}}
\newcommand{\reals}{\mathbb{R}}
\newcommand{\integers}{\mathbb{Z}}
\title{The dimensions of physical quantities}
\author{James Geddes}
\begin{document}
\maketitle

The following is an unusual description of what constitutes
``physics''. Everything is to be understood to be in scare quotes. There will be an
air of apparently unjustified formality about this description. I believe that
the statements made can be replaced with rigorous versions (and proofs adduced)
but I make no claims to have done so.

Fix, once and for all, a set of vector spaces, $U$, $V$, $W$, \ldots. These
vector spaces are the ``home'' of physical quantities. That is, any particular
statment of a physical situation can be described by giving a vector in each of
these vector spaces. (We imagine that they are, for example, the vector spaces
of ``displacement'', ``velocity'', ``energy'', ``force'', and so forth.) Without
loss of generality, imagine that there are only three possible kinds of physical
entity: $U$, $V$, and $W$. That is, a state of the world is a vector in $U\oplus
V\oplus W$.

Now consider the following construction. For any vector space $V$, define
$V^2=V\otimes V$, and likewise for higher ``powers''. Set
\[
\physics = \reals \oplus (U\oplus V\oplus W) \oplus (U\oplus V\oplus W)^2
\oplus \dotsb.
\]
Note that ``$\otimes$'' is distributive over ``$\oplus$''. 

Let $s= u\oplus v\oplus w\in U\oplus V\oplus W$ be some state (not necessarily
one which satisifies the ``laws of physics''). Note that there is a natural
injection $U\times V \to U\otimes V$ (for example) but not $V\to V^*$. (So if
some space $V^*$ is required to describe the physics it will need to be put in
by hand.)

By $\exp(v)$ is meant the element of $\physics$ given by
\[
\exp(V) \equiv 1\oplus
v\oplus (v\otimes v) \oplus \dotsb .
\] 

By a \emph{physical law} is meant a map $L:\physics\to(U\oplus V\oplus\dotsb)$
together with the assertion that, for $v\in U\oplus V\oplus W$ a description of
a possible world, $L(\exp v)=0$. For example, if $U$ is the space of force, $V$ the
space of displacement, and $W$ the space of energy; then the law ``$\text{work}
= \text{force}\times\text{distance}$''

Duals.


\begin{enumerate}
% \item $\reals\in\physics$,
\item For any $V, W\in\physics$ we have $V\otimes W\in\physics$; and
\item For any $V\in\physics$ we have $V^*\in\physics$.
\end{enumerate}

The view here is that we wish to write down physical quantities. The homes for
these physical quantities shall be the vector spaces that are the elements
of~$\physics$: any physical quantity must be a vector in one of the elemets
of~$\\physics$. We shall call an element of $\physics$ a ``quantity-kind''.

There is a natural isomorphism between $V\otimes W$ and $W\otimes V$ so we shall
consider these identical; likewise we shall treat $(V\otimes W)^*$ as identical,
under the natural isomorphism, to $V^*\otimes W^*$. 

Note that $\physics$ almost has a ``vector-space--like'' structure: consider the
operation of taking the tensor product as ``addition'' and taking the dual as
``multiplication by $-1$''. (Strictly speaking, it's not almost a vector space because
only multiplication by the integers is defined. I suppose it's almost a module over the
integers.) It's not quite a vector space (sorry, module over the integers)
though, because there's no ``zero'' vector, and in any case $V\otimes V^*$ isn't it.  

Let $\dimensions$ be a (finite-dimensional!) module over $\integers$ and suppose there is
given, for each $V\in\physics$, a linear map $L(V):V\to\dimensions$ such that:
\begin{enumerate}
\item $L(V\otimes W) = L(V) + L(W)$;
\item $L(V\otimes V^*) = 0$.
\end{enumerate}



  



\end{document}
